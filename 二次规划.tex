\section{二次规划}
\begin{enumerate}
    \item 解下面二次规划问题并且作图解释几何意义.
    \[\begin{array}{lll}
        \max & f(x)=2x_1+3x_2+4x_1^2+2x_1x_2+x_2^2\\
        \mathrm{s.t.} & x_1-x_2 \geqslant 0,\\
        & x_1+x_2 \leqslant 4,\\
        & x_1 \leqslant 3.
    \end{array}\]
    \sol $\displaystyle \max=52,x_1=3,x_2=1.$
    \item 从一点$x_0$的超平面$\{x|Ax=b\}$寻找最短距离能构成二次规划形式,这里$A$是行满秩矩阵
    \[\begin{array}{lll}
        \min & \displaystyle f(x)=\frac{1}{2}(x-x_0)^\mathrm{T}(x-x_0)\\
        \mathrm{s.t.} & Ax=b.
    \end{array}\]
    证明最优乘子为\[\lambda^*=(AA^\mathrm{T})^{-1}(b-Ax_0)\]和最优解为\[x^*=x_0-A^\mathrm{T}(AA^\mathrm{T})^{-1}(b-Ax_0),\]
    并且证明当$A$是行向量时,从$x_0$至$Ax = b$的可行解集的最短距离为\[\frac{|b-Ax_0|}{\|A\|}.\]
    \textcolor{red}{[注意:最优解应为$x^*=x_0+A^\mathrm{T}(AA^\mathrm{T})^{-1}(b-Ax_0)$]}\\
    \pro 令$A$是$m \times n$的矩阵,由于$A$是行满秩矩阵,则$\rank(A)=m$,而$\rank(AA^\mathrm{T})=\rank(A)=m$,则$AA^\mathrm{T}$是$m \times m$的可逆方阵.\\
    本问题的Lagrange函数为$\displaystyle L(x;\lambda)=\frac{1}{2}(x-x_0)^\mathrm{T}(x-x_0)-\lambda^\mathrm{T}(Ax-b)$,则KKT条件为
    \[\begin{cases}
        (x-x_0)=A^\mathrm{T}\lambda\\
        Ax=b
    \end{cases}\]
    于是,$AA^\mathrm{T}\lambda=A(x-x_0)=Ax-Ax_0=b-Ax_0$,而$AA^\mathrm{T}$是$m \times m$的可逆方阵,则$\lambda^*=(AA^\mathrm{T})^{-1}(b-Ax_0)$,再代入第一个KKT条件,得$x^*=x_0+A^\mathrm{T}(AA^\mathrm{T})^{-1}(b-Ax_0)$.\\
    当$A$是行向量,即$m=1$时,
    \begin{align*}
        d^2 = \|x-x_0\|^2 & =(x-x_0)^\mathrm{T}(x-x_0)\\
        &=(b-Ax_0)^\mathrm{T}\cdot \frac{1}{(AA^\mathrm{T})^2}AA^\mathrm{T}(b-Ax_0)\\
        & = \frac{(b-Ax_0)^\mathrm{T}(b-Ax_0)}{AA^\mathrm{T}}
    \end{align*}
    因此,$\displaystyle d=\frac{|b-Ax_0|}{\|A\|}$. 证毕.
    \item 证明(6.2.1)的一阶必要性条件是(6.2.2).\\
    \omitted
    \item 写出下面既有等式约束又有不等式约束的二次凸规划的KKT条件
    \[\begin{array}{lll}
        \min & \displaystyle q(x)=\frac{1}{2}x^\mathrm{T}Gx+x^\mathrm{T}g\\
        \mathrm{s.t.} & Ax \geqslant b,\quad \bar{A}x=\bar{b}.
    \end{array}\]
    这里$G$是对称正半定矩阵,使用这些条件导出类似的原始—对偶步.\\
    \omitted
    \item 考虑二次规划
    \[\begin{array}{lll}
        \min & q(x)=6x_1+4x_2-13-x_1^2-x_2^2\\
        \mathrm{s.t.} & x_1+x_2 \leqslant 3,\\
        & x_1,x_2 \geqslant 0.
    \end{array}\]
    使用图解法求解,再使用有效约束集方法求解.\\
    \sol $\displaystyle \min=-13,x_1=0,x_2=0.$
    \item 用有效集方法求解二次规划问题
    \[\begin{array}{lll}
        \min & q(x)=x_1^2+2x_2^2-2x_1-6x_2-2x_1x_2-2\\
        \mathrm{s.t.} & \displaystyle \frac{1}{2}x_1+\frac{1}{2}x_2 \leqslant 1,\\
        & -x_1+2x_2 \leqslant 2,\\
        & x_1,x_2 \geqslant 0.
    \end{array}\]
    \sol $\displaystyle \min=-\frac{46}{5},x_1=\frac{4}{5},x_2=\frac{6}{5}.$
    \item 推导(6.2.24).\\
    \omitted
\end{enumerate}
\clearpage