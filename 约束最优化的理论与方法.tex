\section{约束最优化的理论与方法}
\begin{enumerate}
    \item 对于$x \in \textbf{R}$,证明函数$\min(z,0)^2$在$z=0$处有二阶不连续导数.\\
    \pro 将这个函数记为$f(z)$,则$f(z)$可写为
    \[f(z) = \begin{cases}
        z^2, & z = 0\\
        0, & z \neq 0
    \end{cases},\quad\therefore
    f''(z)=\begin{cases}
        1, & z \neq 0\\+ \infty, & z\to 0^+\\-\infty, & z\to 0^-
    \end{cases}\]
    所以,函数$\min(z,0)^2$在$z=0$处有二阶不连续导数.
    \item 考虑问题
    \[\begin{array}{lll}
        \min & \displaystyle \frac{1}{1+x^2}\\
        \mathrm{s.t.} & x \geqslant 1,
    \end{array}\]
    写出$P(x;\mu)$.\\
    \sol 对数障碍函数为
    \[P(x;\mu)=\frac{1}{1+x^2}-\mu\log(x-1).\]
    分数障碍函数为
    \[P(x;\mu)=\frac{1}{1+x^2}-\frac{\mu}{x-1}.\]
    \item 考虑问题
    \[\begin{array}{lll}
        \min & x\\
        \mathrm{s.t.} & x^2 \geqslant 0,\\
        & x+1 \geqslant 0,
    \end{array}\]
    解为$x^*=-1$,写出此问题的$P(x;\mu)$,寻找局部极小点.\\
    \sol $P(x;\mu)=x-\mu\log(x^2)-\mu\log(x+1)$,则
    \[\frac{\mathrm{d} P}{\mathrm{d} x}=1-\mu\frac{2x}{x^2}-\mu\frac{1}{x+1}=0,\]
    解得\[x = \frac{3\mu-1 \pm \sqrt{(1-3\mu)^2+8\mu}}{2}(\text{由可行性舍去}-\text{的情况}),\]
    令$\mu \to 0^+$,则$x=-1$.\\
    也可以将$x^2 \geqslant 0$舍弃,因为这是个无效约束,则$P(x;\mu)=x-\mu\log(x+1)$,则
    \[\frac{\mathrm{d} P}{\mathrm{d} x}=1-\mu\frac{1}{x+1}=0,\]
    解得\[x=\mu-1,\]
    令$\mu \to 0^+$,则$x=-1$.
    \item 假设当前点$x_k$是$P(x;\mu_k)$的精确极小点,写出关于当前点$x_k$对于$P(x;\mu_{k+1})$的牛顿步.\\
    \omitted
    \item 试推导(7.4.24).\\
    \omitted
    \item 对附录中试验函数$2.1-2.3$分别用罚函数法和序列二次规划方法求解(用C,MATLAB或FORTRAN等语言编制程序).\\
    \omitted
\end{enumerate}
\clearpage
{\heiti 第七章:补充题目}
\begin{enumerate}
    \item 考虑约束最优化问题
    \[\begin{array}{lll}
        \min & (x_1-2)^2+x_2^2\\
        \mathrm{s.t.} & x_1^2-x_2 \leqslant 0,\\
        & x_1-x_2 \leqslant 0,
    \end{array}\]
    求出其KKT点并判断是否为其全局最优解.\\
    \sol 本问题的KKT条件为
    \[\begin{cases}
        \left(\begin{array}{cc}
            \begin{matrix}
                2x_1-4 \\ 2x_2
            \end{matrix}
        \end{array}\right)=\lambda_1\left(\begin{array}{cc}
            \begin{matrix}
                -2x_1 \\ 1
            \end{matrix}
        \end{array}\right)+\lambda_2\left(\begin{array}{cc}
            \begin{matrix}
                -1 \\ 1
            \end{matrix}
        \end{array}\right)\\
        \lambda_1 \geqslant 0, \lambda_2 \geqslant 0,\\
        \lambda_1(x_1^2-x_2)=0,\lambda_2(x_1-x_2)=0.
    \end{cases}\]
    利用第三个KKT条件,可得以下情况:
    \begin{enumerate}[label=(\arabic*)]
        \item $\lambda_1=0,\lambda_2=0,x_1=2,x_2=0$,不满足约束条件,不是KKT点.
        \item $\lambda_1=0,x_1-x_2=0 \Rightarrow x_1=x_2=1,\lambda_2=2$,是KKT点.
        \item $x_1^2-x_2=0,\lambda_2=0 \Rightarrow 2x_1^3+x_1-2=0$,令$g(x_1)=2x_1^3+x_1-2$,显然$g(x_1)$单调递增,而$g(0)=-2<0,g(1)=1>0$,于是$x_1 \in (0,1),x_2=x_1^2 < x_1$,不满足第二个约束,不是KKT点.
        \item $x_1^2-x_2=0,x_1=x_2=0 \Rightarrow x^{(1)}=(0,0)^\mathrm{T},x^{(2)}=(1,1)^\mathrm{T}$,$x^{(2)}$就是情况(2),$x^{(1)}$对应的$\lambda_1=-4,\lambda_2=4$不满足KKT条件,不是KKT点.
    \end{enumerate}
    由于这个问题是个凸规划,所以其KKT点就是全局最优解.
    \item 考虑约束最优化问题
    \[\begin{array}{lll}
        \min & x_1^2+2x_2^2\\
        \mathrm{s.t.} & x_1+x_2-1 \geqslant 0,
    \end{array}\]
    利用对数障碍函数法求解.\\
    \sol 对数障碍函数$P(x;\mu)=x_1^2+2x_2^2-\mu\log(x_1+x_2-1)$,则
    \[\begin{cases}
        \displaystyle \frac{\partial P}{\partial x_1}=2x_1-\frac{\mu}{x_1+x_2-1}=0\\
        \displaystyle \frac{\partial P}{\partial x_2}=4x_2-\frac{\mu}{x_1+x_2-1}=0
    \end{cases}\]
    解得
    \[\left(\begin{array}{cc}
        \begin{matrix}
            x_1 \\ x_2
        \end{matrix}
    \end{array}\right)=\left(\begin{array}{cc}
        \begin{matrix}
            \displaystyle \frac{1+\sqrt{1+3\mu}}{3} \\ \displaystyle \frac{1+\sqrt{1+3\mu}}{6}
        \end{matrix}
    \end{array}\right),\quad\left(\begin{array}{cc}
        \begin{matrix}
            x_1 \\ x_2
        \end{matrix}
    \end{array}\right)=\left(\begin{array}{cc}
        \begin{matrix}
            \displaystyle \frac{1-\sqrt{1+3\mu}}{3} \\ \displaystyle \frac{1-\sqrt{1+3\mu}}{6}
        \end{matrix}
    \end{array}\right)(\text{不可行,舍去})\]
    当$\mu \to 0^+$时,\[\left(\begin{array}{cc}
        \begin{matrix}
            x_1 \\ x_2
        \end{matrix}
    \end{array}\right)=\left(\begin{array}{cc}
        \begin{matrix}
            \displaystyle \frac{2}{3} \\ \displaystyle \frac{1}{3}
        \end{matrix}
    \end{array}\right).\]
\end{enumerate}