\section{线性搜索与信赖域方法}
\begin{enumerate}
    \item 推导0.618法的迭代公式.\\
    \sol 由Fibonacci法迭代公式:
    \[\lambda_k = a_k + \left(1 - \frac{F_{n-k}}{F_{n-k+1}}\right)(b_k - a_k), \quad \mu_k = a_k + \frac{F_{n-k}}{F_{n-k+1}} (b_k - a_k),\]
    由式(3.2.20)可知:$\displaystyle \tau = \lim_{n \to \infty} \frac{F_{n-k}}{F_{n-k+1}} = \frac{\sqrt{5}-1}{2}$,取$\tau \approx 0.618$,则可得0.618法的迭代公式:
    \[\lambda_k = a_k + 0.382(b_k - a_k), \quad \mu_k = a_k + 0.618(b_k - a_k).\]
    \item 分别用0.618法,二次插值法,Goldstein不精确线性搜索方法求函数
    \[f(x)=(\sin x)^6 \tan (1-x) \mathrm{e}^{30x}\]在区间$[0,1]$上的极大值.\\
    \sol 由于这三种方法都是求解函数的极小值,故我们不妨设
    \[g(x)=-(\sin x)^6 \tan (1-x) \mathrm{e}^{30x}.\]
    0.618法:(取精度为0.0001)
    \begin{table}[H]
        \centering
        \begin{tabular}{|c|c|c|c|c|}
            \hline
            迭代次数$k$ & $[a_k,b_k]$ & $[\lambda_k,\mu_k]$ & $g(\lambda_k)$ & $g(\mu_k)$ \\ \hline
            $0$ & $[0.00000,1.00000]$ & $[0.38200,0.61800]$ & $-180.93169$ & $-1712603.84319$\\ \hline
            $1$ & $[0.38200,1.00000]$ & $[0.61800,0.76392]$ & $-1712603.84319$ & $-236591928.32087$\\ \hline
            $2$ & $[0.61800,1.00000]$ & $[0.76392,0.85408]$ & $-236591928.32087$ & $-3621963454.74789$\\ \hline
            $3$ & $[0.76392,1.00000]$ & $[0.85408,0.90982]$ & $-3621963454.74789$ & $-15629095526.75970$\\ \hline
            $4$ & $[0.85408,1.00000]$ & $[0.90982,0.94426]$ & $-15629095526.75970$ & $-31645853714.38405$\\ \hline
            $5$ & $[0.90982,1.00000]$ & $[0.94426,0.96555]$ & $-31645853714.38405$ & $-40525854255.03307$\\ \hline
            $6$ & $[0.94426,1.00000]$ & $[0.96555,0.97871]$ & $-40525854255.03307$ & $-39217975674.21036$\\ \hline
            $7$ & $[0.94426,0.97871]$ & $[0.95742,0.96555]$ & $-37941310649.54459$ & $-40525854255.03307$\\ \hline
            $8$ & $[0.95742,0.97871]$ & $[0.96555,0.97057]$ & $-40525854255.03307$ & $-41085789962.33502$\\ \hline
            $9$ & $[0.96555,0.97871]$ & $[0.97057,0.97368]$ & $-41085789962.33502$ & $-40851559012.30843$\\ \hline
            $10$ & $[0.96555,0.97368]$ & $[0.96866,0.97057]$ & $-40993501131.26535$ & $-41085789962.33502$\\ \hline
            $11$ & $[0.96866,0.97368]$ & $[0.97057,0.97176]$ & $-41085789962.33502$ & $-41056235078.63431$\\ \hline
            $12$ & $[0.96866,0.97176]$ & $[0.96984,0.97057]$ & $-41070108314.89887$ & $-41085789962.33502$\\ \hline
            $13$ & $[0.96984,0.97176]$ & $[0.97057,0.97103]$ & $-41085789962.33502$ & $-41082739735.01510$\\ \hline
            $14$ & $[0.96984,0.97103]$ & $[0.97030,0.97057]$ & $-41082767355.81810$ & $-41085789962.33502$\\ \hline
            $15$ & $[0.97030,0.97103]$ & $[0.97057,0.97075]$ & $-41085789962.33502$ & $-41085803940.89211$\\ \hline
            $16$ & $[0.97057,0.97103]$ & $[0.97075,0.97085]$ & $-41085803940.89211$ & $-41085091885.42854$\\ \hline
            $17$ & $[0.97057,0.97085]$ & $[0.97068,0.97075]$ & $-41085973087.67970$ & $-41085803940.89211$\\ \hline
        \end{tabular}
    \end{table}
    故$f(x)$的极大点为0.97071,极大值为$-4.1086\times10^{10}$.\\
    二次插值法:(取精度为0.0001,初始迭代点为0.1、0.5、1)
    {\small
    \begin{table}[H]
        \centering
        \begin{tabular}{|c|c|c|c|c|c|c|c|c|}
            \hline
            $k$ & $a_1$ & $a_2$ & $a_3$ & $\bar a$ & $g_1$ & $g_2$ & $g_3$ & $\bar g$ \\ \hline
            $0$ & $0.10000$ & $0.50000$ & $1$ & $0.55000$ & $-0.00003$ & $-21685.89733$ & $0$ & $-144313.48798$ \\ \hline
            $1$ & $0.50000$ & $0.55000$ & $1$ & $0.74609$ & $-21685.89733$ & $-144313.48798$ & $0$ & $-133409650.84449$ \\ \hline
            $2$ & $0.55000$ & $0.74609$ & $1$ & $0.77494$ & $-144313.48798$ & $-133409650.84449$ & $0$ & $-335472520.81192$ \\ \hline
            $3$ & $0.74609$ & $0.77494$ & $1$ & $0.86519$ & $-133409650.84449$ & $-335472520.81192$ & $0$ & $-4941583365.98999$ \\ \hline
$4$ & $0.77494$ & $0.86519$ & $1$ & $0.88556$ & $-335472520.81192$ & $-4941583365.98999$ & $0$ & $-8543117362.08340$ \\ \hline
$5$ & $0.86519$ & $0.88556$ & $1$ & $0.92277$ & $-4941583365.98999$ & $-8543117362.08340$ & $0$ & $-20938281955.02223$ \\ \hline
$6$ & $0.88556$ & $0.92277$ & $1$ & $0.93571$ & $-8543117362.08340$ & $-20938281955.02223$ & $0$ & $-27213528026.34990$ \\ \hline
$7$ & $0.92277$ & $0.93571$ & $1$ & $0.94986$ & $-20938281955.02223$ & $-27213528026.34990$ & $0$ & $-34493283849.16349$ \\ \hline
$8$ & $0.93571$ & $0.94986$ & $1$ & $0.95654$ & $-27213528026.34990$ & $-34493283849.16349$ & $0$ & $-37577724676.75349$ \\ \hline
$9$ & $0.94986$ & $0.95654$ & $1$ & $0.96193$ & $-34493283849.16349$ & $-37577724676.75349$ & $0$ & $-39577028005.21829$ \\ \hline
$10$ & $0.95654$ & $0.96193$ & $1$ & $0.96495$ & $-37577724676.75349$ & $-39577028005.21829$ & $0$ & $-40395052418.74240$ \\ \hline
$11$ & $0.96193$ & $0.96495$ & $1$ & $0.96706$ & $-39577028005.21829$ & $-40395052418.74240$ & $0$ & $-40798680632.87956$ \\ \hline
$12$ & $0.96495$ & $0.96706$ & $1$ & $0.96834$ & $-40395052418.74240$ & $-40798680632.87956$ & $0$ & $-40963298550.04148$ \\ \hline
$13$ & $0.96706$ & $0.96834$ & $1$ & $0.96919$ & $-40798680632.87956$ & $-40963298550.04148$ & $0$ & $-41035621691.04285$ \\ \hline
$14$ & $0.96834$ & $0.96919$ & $1$ & $0.96972$ & $-40963298550.04148$ & $-41035621691.04285$ & $0$ & $-41065099592.32568$ \\ \hline
$15$ & $0.96919$ & $0.96972$ & $1$ & $0.97006$ & $-41035621691.04285$ & $-41065099592.32568$ & $0$ & $-41077456061.09961$ \\ \hline
$16$ & $0.96972$ & $0.97006$ & $1$ & $0.97028$ & $-41065099592.32568$ & $-41077456061.09961$ & $0$ & $-41082487976.77145$ \\ \hline
$17$ & $0.97006$ & $0.97028$ & $1$ & $0.97042$ & $-41077456061.09961$ & $-41082487976.77145$ & $0$ & $-41084558359.15994$ \\ \hline
$18$ & $0.97028$ & $0.97042$ & $1$ & $0.97051$ & $-41082487976.77145$ & $-41084558359.15994$ & $0$ & $-41085400853.61729$ \\ \hline
        \end{tabular}
    \end{table}}
    故$f(x)$的极大点为0.97051,极大值为$-4.1085\times10^{10}$.\\
    Goldstein不精确线性搜索方法略.
    \item 写出Fibonacci法的计算过程和C程序(或MATLAB,FORTRAN程序).\\
    \sol Fibonacci法的计算过程:
    \begin{enumerate}[label=步\arabic*:,left=0em]
        \item 给定初始搜索区间$[a_1,b_1]$和精度要求$\delta > 0$,计算满足$\displaystyle F_n \geqslant \frac{b_1-a_1}{\delta}$的最小的$n$;
        \item $\displaystyle \lambda_1 := a_1 + \frac{F_{n-2}}{F_n}(b_1-a_1)$,$\displaystyle \mu_1 := a_1 + \frac{F_{n-1}}{F_n}(b_1-a_1)$\\
        计算$\varphi(\lambda_1),\varphi(\mu_1)$,置$k:=1$;
        \item 若$k = n - 1$,则转步4,否则转步5;
        \item 停止计算,输出$\lambda_k$(或$\mu_k$,因为$\lambda_k = \mu_k$此时恰好成立). 也可再计算一个函数值$\displaystyle \varphi \left(\frac{a_k+b_k}{2}\pm\varepsilon\right)$,比较后输出$\lambda_k$或$\displaystyle \frac{a_k+b_k}{2}\pm\varepsilon$;
        \item 若$\varphi(\lambda_k) > \varphi(\mu_k)$,转步6,否则转步7;
        \item 令$\displaystyle a_{k+1} := \lambda_k, b_{k+1} := b_k, \lambda_{k+1} := \mu_k, \varphi(\lambda_{k+1}) := \varphi(\mu_k), \mu_{k+1} := a_{k+1} + \frac{F_{n-k-1}}{F_{n-k}}(b_{k+1}-a_{k+1})$,计算$\varphi(\mu_{k+1})$,令$k := k+1$,转步3;
        \item 令$\displaystyle a_{k+1} := a_k, b_{k+1} := \mu_k, \mu_{k+1} := \lambda_k, \varphi(\mu_{k+1}) := \varphi(\lambda_k), \lambda_{k+1} := a_{k+1} + \frac{F_{n-k-2}}{F_{n-k}}(b_{k+1}-a_{k+1})$,计算$\varphi(\lambda_{k+1})$,令$k := k+1$,转步3.
    \end{enumerate}
    注:在步6、步7中,当$k = n - 2$,无需计算$\varphi(\mu_{k+1})$,$\varphi(\lambda_{k+1})$. 因为此时$\mu_{k+1} = \lambda_{k+1}$,但由于舍入误差,它们未必精确相等.\\
    Matlab程序为:
    \begin{lstlisting}
function [xmin, fval, k] = Fibonacci(f, a, b, delta)
    % f是所求函数表达式,a是区间左端点
    % b是区间右端点,delta是精度
    % xmin是极小点,fval是极小值,k是迭代步数
    format long
    Fn = (b - a) / delta; F = []; F(1) = 1; F(2) = 1; i = 1;
    while F(i) < Fn
        F(i + 2) = F(i) + F(i + 1); i = i + 1;
    end
    n = i;
    lambda = a + F(n - 2) / F(n) * (b - a); mu = a + F(n - 1) / F(n) * (b - a);
    x = symvar(f);
    phi_lambda = subs(f, x, lambda); phi_mu = subs(f, x, mu);
    k = 0;
    fprintf('k \t [a_k, b_k] \t [lambda_k, mu_k] \t phi(lambda_k) \t phi(mu_k)\n');
    fprintf('%d \t [%.5f,%.5f] \t [%.5f,%.5f] \t %.5f \t %.5f\n', k, a, b, lambda, mu, phi_lambda, phi_mu);
    k = 1;
    while true
        if k == n - 1
            xmin = vpa(lambda, 5); fval = vpa(subs(f, x, lambda), 5);
            break;
        else
            if phi_lambda > phi_mu
                a = lambda; b = b; lambda = mu; phi_lambda = phi_mu;
                if k == n - 2
                    xmin = vpa(lambda, 5);
                    fval = vpa(subs(f, x, lambda), 5);
                    break;
                else
                    mu = a + F(n - k - 1) / F(n - k) * (b - a);
                    phi_mu = subs(f, x, mu); k = k + 1;
                    fprintf('%d \t [%.5f,%.5f] \t [%.5f,%.5f] \t %.5f \t %.5f\n', k, a, b, lambda, mu, phi_lambda, phi_mu);
                end
            else
                a = a; b = mu; mu = lambda; phi_mu = phi_lambda;
                if k == n - 2
                    xmin = vpa(lambda, 5);
                    fval = vpa(subs(f, x, lambda), 5);
                    break;
                else
                    lambda = a + F(n - k - 2) / F(n - k) * (b - a);
                    phi_lambda = subs(f, x, lambda); k = k + 1;
                    fprintf('%d \t [%.5f,%.5f] \t [%.5f,%.5f] \t %.5f \t %.5f\n', k, a, b, lambda, mu, phi_lambda, phi_mu);
                end
            end
        end
    end
end
    \end{lstlisting}
    \item 设$\varphi(t)=\mathrm{e}^{-t}+\mathrm{e}^t$,区间为$[-1,1]$.
    \begin{enumerate}[label=(\arabic*)]
        \item 用0.618法极小化$\varphi(t)$.
        \item 用Fibonacci法极小化$\varphi(t)$.
    \end{enumerate}
    \sol \begin{enumerate}[label=(\arabic*)]
        \item 取精度为0.01:
    \begin{table}[H]
        \centering
        \begin{tabular}{|c|c|c|c|c|}
            \hline
            迭代次数$k$ & $[a_k,b_k]$ & $[\lambda_k,\mu_k]$ & $\varphi(\lambda_k)$ & $\varphi(\mu_k)$ \\ \hline
            $0$ & $[-1.00000,1.00000]$ & $[-0.23600,0.23600]$ & $2.05595$ & $2.05595$\\ \hline
            $1$ & $[-1.00000,0.23600]$ & $[-0.52785,-0.23600]$ & $2.28515$ & $2.05595$\\ \hline
            $2$ & $[-0.52785,0.23600]$ & $[-0.23600,-0.05579]$ & $2.05595$ & $2.00311$\\ \hline
            $3$ & $[-0.23600,0.23600]$ & $[-0.05579,0.05570]$ & $2.00311$ & $2.00310$\\ \hline
            $4$ & $[-0.05579,0.23600]$ & $[0.05570,0.12454]$ & $2.00310$ & $2.01553$\\ \hline
            $5$ & $[-0.05579,0.12454]$ & $[0.01309,0.05570]$ & $2.00017$ & $2.00310$\\ \hline
            $6$ & $[-0.05579,0.05570]$ & $[-0.01320,0.01309]$ & $2.00017$ & $2.00017$\\ \hline
            $7$ & $[-0.01320,0.05570]$ & $[0.01309,0.02938]$ & $2.00017$ & $2.00086$\\ \hline
            $8$ & $[-0.01320,0.02938]$ & $[0.00306,0.01309]$ & $2.00001$ & $2.00017$\\ \hline
            $9$ & $[-0.01320,0.01309]$ & $[-0.00316,0.00306]$ & $2.00001$ & $2.00001$\\ \hline
        \end{tabular}
    \end{table}
    故$\varphi(x)$的极小点为$-0.000046968$,极小值为2.
        \item 取精度为0.01:
        \begin{table}[H]
            \centering
            \begin{tabular}{|c|c|c|c|c|}
                \hline
                迭代次数$k$ & $[a_k,b_k]$ & $[\lambda_k,\mu_k]$ & $\varphi(\lambda_k)$ & $\varphi(\mu_k)$ \\ \hline
                $0$ & $[-1.00000,1.00000]$ & $[-0.23605,0.23605]$ & $2.05598$ & $2.05598$\\ \hline
                $2$ & $[-1.00000,0.23605]$ & $[-0.52790,-0.23605]$ & $2.28521$ & $2.05598$\\ \hline
                $3$ & $[-0.52790,0.23605]$ & $[-0.23605,-0.05579]$ & $2.05598$ & $2.00311$\\ \hline
                $4$ & $[-0.23605,0.23605]$ & $[-0.05579,0.05579]$ & $2.00311$ & $2.00311$\\ \hline
                $5$ & $[-0.23605,0.05579]$ & $[-0.12446,-0.05579]$ & $2.01551$ & $2.00311$\\ \hline
                $6$ & $[-0.12446,0.05579]$ & $[-0.05579,-0.01288]$ & $2.00311$ & $2.00017$\\ \hline
                $7$ & $[-0.05579,0.05579]$ & $[-0.01288,0.01288]$ & $2.00017$ & $2.00017$\\ \hline
                $8$ & $[-0.05579,0.01288]$ & $[-0.03004,-0.01288]$ & $2.00090$ & $2.00017$\\ \hline
                $9$ & $[-0.03004,0.01288]$ & $[-0.01288,-0.00429]$ & $2.00017$ & $2.00002$\\ \hline
                $10$ & $[-0.01288,0.01288]$ & $[-0.00429,0.00429]$ & $2.00002$ & $2.00002$\\ \hline
                $11$ & $[-0.01288,0.00429]$ & $[-0.00429,-0.00429]$ & $2.00002$ & $2.00002$\\ \hline
            \end{tabular}
        \end{table}
        故$\varphi(x)$的极小点为$-0.0042918$,极小值为2.
    \end{enumerate}
    \item 用二分法解
    \[\begin{array}{lll}
        \min & x^2 + 2x\\
        \mathrm{s.t.} & -3 \leqslant x \leqslant 6.
    \end{array}\]
    取最后区间长度为$\delta=0.2$. (解$x^*=-0.961$).\\
    \sol \begin{table}[H]
        \centering
        \begin{tabular}{|c|c|}
            \hline
            $k$ & $[a_k, b_k]$ \\ \hline
            $1$ & $[-3.00000,6.00000]$ \\ \hline
            $2$ & $[-3.00000,1.50000]$ \\ \hline
            $3$ & $[-3.00000,-0.75000]$ \\ \hline
            $4$ & $[-1.87500,-0.75000]$ \\ \hline
            $5$ & $[-1.31250,-0.75000]$ \\ \hline
            $6$ & $[-1.03125,-0.75000]$ \\ \hline
            $7$ & $[-1.03125,-0.89063]$ \\ \hline
        \end{tabular}
    \end{table}
    故$x^2 + 2x$在$[-3,6]$上的极小点为$-0.96094$,极小值为$-0.99847$.
    \item 设$\varphi(t)=1-t\mathrm{e}^{-t^2}$,区间为$[0,1]$. 试用三点二次插值法极小化$\varphi(t)$.\\
    \sol 取精度为0.0001,初始迭代点为0、0.5、1
    {\small
    \begin{table}[H]
        \centering
        \begin{tabular}{|c|c|c|c|c|c|c|c|c|}
            \hline
            $k$ & $a_1$ & $a_2$ & $a_3$ & $\bar a$ & $\varphi_1$ & $\varphi_2$ & $\varphi_3$ & $\bar \varphi$ \\ \hline
            $0$ & $0.00000$ & $0.50000$ & $1.00000$ & $0.72381$ & $1.00000$ & $0.61060$ & $0.63212$ & $0.57136$ \\ \hline
            $1$ & $0.50000$ & $0.72381$ & $1.00000$ & $0.72278$ & $0.61060$ & $0.57136$ & $0.63212$ & $0.57133$ \\ \hline
            $2$ & $0.50000$ & $0.72278$ & $0.72381$ & $0.70822$ & $0.61060$ & $0.57133$ & $0.57136$ & $0.57112$ \\ \hline
            $3$ & $0.50000$ & $0.70822$ & $0.72278$ & $0.70769$ & $0.61060$ & $0.57112$ & $0.57133$ & $0.57112$ \\ \hline
            $4$ & $0.50000$ & $0.70769$ & $0.70822$ & $0.70716$ & $0.61060$ & $0.57112$ & $0.57112$ & $0.57112$ \\ \hline
            $5$ & $0.50000$ & $0.70716$ & $0.70769$ & $0.70713$ & $0.61060$ & $0.57112$ & $0.57112$ & $0.57112$ \\ \hline
        \end{tabular}
    \end{table}}
    故$\varphi(t)$的极小点为0.70713,极小值为0.57112.
    \item 设$\varphi(t)=-2t^3+21t^2-60t+50$.
    \begin{enumerate}[label=(\arabic*)]
        \item 用Goldstein方法极小化$\varphi(t),t_0=0.5,\rho=0.1$.
        \item 用Wolfe方法极小化$\varphi(t),t_0=0.5,\rho=0.1,\sigma=0.8$.
    \end{enumerate}
    \omitted
    \item 设$f(x)=x_1^4+x_2^2+x_2^2$,给定当前点$x_k=(1,1)^\mathrm{T}$,方向$d_k=(-3,-1)^\mathrm{T}$,并设$\rho=1,t=0.5$.
    \begin{enumerate}[label=(\arabic*)]
        \item 试分别用Goldstein方法和Wolfe方法求新点$x_{k+1}$.
        \item 分别以$\alpha=1,\alpha=0.5,\alpha=0.1$说明哪些$\alpha$满足Wolfe准则,哪些$\alpha$不满足Wolfe准则.
    \end{enumerate}
    \omitted
    \item 设$\displaystyle f(x)=\frac{1}{2}x_1^2+x_2^2$. 设初始点$x_0=(1,1)^\mathrm{T}$. 对于(1)$\Delta_0=1$, (2)$\displaystyle\Delta_0=\frac{5}{4}$,
    \begin{enumerate}[label=(\roman*)]
        \item 用折线法计算下一个迭代点$x_1$;
        \item 用双折线法计算下一个迭代点$x_1$.
    \end{enumerate}
    \omitted
    \item 证明:
    \begin{enumerate}[label=(\arabic*)]
        \item Cauchy点满足(3.8.4).
        \item 子问题(3.7.2)的精确极小点满足(3.8.4).
    \end{enumerate}
    \omitted
\end{enumerate}
\clearpage